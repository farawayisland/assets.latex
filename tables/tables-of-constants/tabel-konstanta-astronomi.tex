% ~/Library/texmf/tex/assets/tables/tables-of-constants/tabel-konstanta-astronomi.tex
\renewcommand*{\raggedchapter}{\centering}
\addchap{Daftar Konstanta}
\renewcommand*{\raggedchapter}{\raggedright}
\begin{table}[H]
  \centering
  \renewcommand{\arraystretch}{1.3}
  \setlength{\tabcolsep}{0.35em}
  \begin{tabular}{|l|c|r|}
    \hline
    \multicolumn{1}{|c|}{\cellcolor{lightgray}\textbf{Nama Besaran}} &
    \multicolumn{1}{c}{\cellcolor{lightgray}\textbf{Notasi}}
    &
    \multicolumn{1}{|c|}{\cellcolor{lightgray}\textbf{Harga}} \\
    \hline
    Satuan astronomi (\textit{astronomical unit}) & \unit{\astronomicalunit} &
    \qty{1.496e11}{\meter} \\
    \hline
    Tahun cahaya (\textit{light-year}) & \unit{\lightyear} &
    \qty{9.461e15}{\meter} \\
    \hline
    Parsek (\textit{parsec}) & \unit{\parsec} & \qty{3.086e16}{\meter} \\
    \hline
    Tahun sideris & & \qty{365.2564}{\day} \\
    \hline
    Tahun tropik & & \qty{365.2422}{\day} \\
    \hline
    Tahun Gregorian & & \qty{365.2425}{\day} \\
    \hline
    Tahun Julian & & \qty{365.25}{\day} \\
    \hline
    Periode sinodis Bulan (\textit{synodic month}) & & \qty{29.5306}{\day} \\
    \hline
    Periode sideris Bulan (\textit{sidereal month}) & & \qty{27.3217}{\day} \\
    \hline
    Hari Matahari rerata (\textit{mean solar day}) & &
    \timeperiod{24;3;56.56} \\
    \hline
    Hari sideris rerata (\textit{mean sidereal day}) & &
    \timeperiod{23;56;4.09} \\
    \hline
    Massa Matahari & \unit{\solarmass} & \qty{1.988e30}{\kilo\gram} \\
    \hline
    Radius Matahari & \unit{\solarradius} & \qty{6.957e8}{\meter} \\
    \hline
    Temperatur Permukaan Matahari & \unit{\solareffectivetemperature} &
    \qty{5772}{\kelvin} \\
    \hline
    Luminositas Matahari & \unit{\solarluminosity} & \qty{3.828e26}{\watt} \\
    \hline
    Fluks Terima Matahari (\textit{solar irradiance}) &
    \unit{\solarirradiance} & \qty{1,361}{\watt\per\square\meter} \\
    \hline
    Magnitudo semu visual Matahari &
    \unit{\solarapparentvisualmagnitude} & \num{-26.78} \\
    \hline
    Magnitudo semu bolometrik Matahari &
    \unit{\solarapparentbolometricmagnitude} & \num{-26.83}\\
    \hline
    \multirow{2}{*}{Indeks warna Matahari} &
    \((B - V)_\Sun\) & \num{0.65} \\
    \cline{2-3} &
    \((U - B)_\Sun\) & \num{0.10} \\
    \hline
    Magnitudo mutlak visual Matahari & \unit{\solarabsolutevisualmagnitude}
    & \num{4.83} \\
    \hline
    Magnitudo mutlak biru Matahari & \unit{\solarabsolutebluemagnitude} &
    \num{5.48} \\
    \hline
    Magnitudo mutlak bolometrik Matahari &
    \unit{\solarabsolutebolometricmagnitude} & \num{4.72}\\
    \hline
    Massa Bulan & \unit{\lunarmass} & \qty{7.348e22}{\kilo\gram} \\
    \hline
    Radius Bulan   & \unit{\lunarradius} & \qty{1.738e6}{\meter} \\
    \hline
    Jarak rerata Bumi–Bulan & & \qty{3.844e8}{\meter} \\
    \hline
    Konstanta Hubble & \unit{\hubbleconstant} &
    \qty{69.3}{\kilo\meter\per\second\per\mega\parsec} \\
    \hline
    Jansky & \unit{\jansky} &
    \qty{e-26}{\watt\per\square\meter\per\hertz} \\
    \hline
  \end{tabular}
  \caption{Konstanta Astronomi}
  % \label{tab:1.1}
\end{table}
