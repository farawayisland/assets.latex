% ~/Library/texmf/tex/assets/tables/tables-of-constants/tabel-konstanta-fisika.tex
\renewcommand{\arraystretch}{1.3}
\setlength{\tabcolsep}{0.35em}
\begin{tabular}{|l|c|r|}
  \hline
  \multicolumn{1}{|c|}{\cellcolor{lightgray}\textbf{Nama Konstanta}} &
  \multicolumn{1}{c}{\cellcolor{lightgray}\textbf{Simbol}}
  &
  \multicolumn{1}{|c|}{\cellcolor{lightgray}\textbf{Harga}} \\
  \hline
  Kecepatan cahaya & \unit{\speedoflight} &
  \qty[group-separator = {}]{2.99792458e8}{\meter\per\second} \\
  \hline
  Konstanta gravitasi & \unit{\gravitationalconstant} &
  \qty{6.674e-11}{\cubic\meter\per\kilogram\per\square\second} \\
  \hline
  Konstanta Planck & \unit{\planckconstant} &
  \qty{6.6261e-34}{\joule\per\second} \\
  \hline
  Konstanta Boltzmann & \unit{\boltzmannconstant} &
  \qty{1.3807e-23}{\joule\per\kelvin} \\
  \hline
  Konstanta Coulomb & \unit{\coulombconstant} &
  \qty{8.988e9}{\newton\square\meter\per\square\coulomb} \\
  \hline
  Konstanta kerapatan radiasi & \unit{\radiationdensityconstant} &
  \qty{7.5657e-6}{\joule\per\cubic\meter\per\kelvin\tothe{4}} \\
  \hline
  Konstanta Stefan–Boltzmann & \unit{\stefanboltzmannconstant} &
  \qty{5.6704e-8}{\watt\per\square\meter\per\kelvin\tothe{4}} \\
  \hline
  Muatan elementer & \unit{\elementarycharge} & \qty{1.6022e-19}{C} \\
  \hline
  Massa elektron & \unit{\electronmass} & \qty{9.1094e-31}{\kilo\gram} \\
  \hline
  Dalton (satuan massa atom) & \unit{\dalton} atau
  \unit{\atomicmassunit} & \qty{1.6605e-27}{\kilo\gram} \\
  \hline
  Massa proton & \unit{\protonmass} & \qty{1.6726e-27}{\kilo\gram} \\
  \hline
  Massa neutron & \unit{\neutronmass} & \qty{1.6749e-27}{\kilo\gram} \\
  \hline
  Massa atom \(\ce{^1_1H}\) & \unit{\protiummass} &
  \qty{1.6735e-27}{\kilo\gram} \\
  \hline
  Massa atom \(\ce{^4_2He}\) & \unit{\atomicmassofheliumfour} &
  \qty{6.6465e-27}{\kilo\gram} \\
  \hline
  Massa inti \(\ce{^4_2He}\) & \unit{\alphaparticlemass} &
  \qty{6.6447e-27}{\kilo\gram} \\
  \hline
  Konstanta gas molar & \unit{\molargasconstant} &
  \qty{8.3145}{\joule\per\kelvin\per\mole} \\
  \hline
\end{tabular}
\caption{Konstanta Fisika}
% \label{tab:1.2}
